\startcomponent comp-lab-week01

\project proj-course_docs

\product prod-76_n87

\environment env-76_n87

\environment env-labnotes

\setvariables[local]
        [short_title={Lab01},
        title={Setting Up},
        subtitle={},
        type={Individual}, 
        submit={to GitHub},  
        id=s19lab1,
        version=001,]


% \startbuffer[abstract]
% The abstract of this thing.
% \stopbuffer

\startchapter[title={Setting Up}]

% \startsubject[title={Agenda}]
% 
% \placecontent
% 
% \stopsubject

\startsection[title={Welcome!}]

Welcome to Web Design Lab!

\stopsection

\startsection[title={Course Goals}]

The primary goal of this course is to give you the knowledge and tools to express design ideas on a web page. We will mainly be concerned with HTML and CSS; while we may take a glancing look at using JavaScript programming to dynamically enhance our web pages, the focus will be on using the core technologies of HTML and CSS, and keeping the implementations simple and the designs clear. Our goal is not to create a fully-formed website, but instead a high definition prototype: it might not have full functionality, but it should look like it could.

This course strives to provide you with a grounding in the foundational technologies that drive the web: HTML to structure text, and CSS to style it. Whether you're coding pages by hand or using a complicated CMS to manage information, everything on the web needs to use these technologies to communicate.

By the end of this class, you should be comfortable with:   

\startitemize[packed]
\item Reading and writing valid, well-formed, and semantic HTML
\item How CSS is used to manipulate the appearance of content
\item The separation of content and presentation, and why this is important
\item Debugging webpages
\item Articulating the reasoning behind your design and interaction choices
\stopitemize

I also hope that working with this stuff ends up being somewhat fun!
\stopsection

\startsection[title={Laptops & Lab Computers}]

Bring your laptop to every class! Part of getting comfortable with this material is gaining confidence with the tools, and it's easier to become familiar with your tools in a consistant environment. Bringing your own machine to each class should also ensure that you have all of the things you need to participate and make progress: tools, code, assets, notes, bookmarks of interesting websites, etc. 

\stopsection

\startsection[title={Tools & Software}]

We will be using the following software for this class:  

\startitemize[packed]
	\item Brackets (or other text editor)
	\item Google Chrome
	\item Git, GitHub, and the GitHub Desktop client
	\item Adobe Photoshop (or other photo editor)
\stopitemize

\startsubsection[title={Brackets}]

{\em Brackets} is an open-source text editor built to develop with HTML, CSS and other web technologies. Brackets is available for MacOS, Windows, and Linux. Brackets is open source, and free to use.

\useURL[brackets][http://brackets.io]
Brackets is available as a free download from: 

\startnarrower
\from[brackets]
\stopnarrower

To create our webpages, we need to use a \keyword{text-editor} to make the HTML and CSS files. A text editor is different from other kinds of editors, like word processors (Microsoft Word, Apple Pages, etc.) and layout software (InDesign, Quark Express, etc.). A text editor stores its work in a \keyword{plain text} format, a universally-standard (as well as arbitrary) way of encoding text in a way that computers can manipulate. Most word processing and layout software store their work in a \keyword{binary} format, to account for extra information about the text: formatting (fonts, font sizes, font weights, etc.), layout (position, margins, etc.), and other non-textual parts of a document. 

\placefigure
    [force]                                       % figure placement
    [fig:plain_vs_binary]                           % figure reference
    {Bob Ross Lorem Ipsum in plain text format and binary format.} % figure caption
    {
    \framed[strut=no,frame=off,width=7in,align=middle,loffset=1in]{
      \startcombination[2*1]                      % contents of the figure: a combination
          {\framed[align=right,bottom=\vss,frame=off,strut=no,width=2in]{%
            \starttyping
We must be quiet, soft and gentle. With practice comes confidence. From all of us here, I want to wish you happy painting and God bless, my friends. It's amazing what you can do with a little love in your heart. 
            \stoptyping}
          } {Plain Text Format}  
          {\framed[align=right,bottom=\vss,frame=off,strut=no,width=2in]{%
            \starttyping
<D0><CF>^Q^Z<E1> ^@^@^@^@^@^@^@^@ ^@^@^@^@^@^@^@ ^@>^@^C^@<FE><FF>        ^@^F^@^@^@^@^@^@ ^@^@^@^@^@^A^@^@^@' ^@^@^@^@^@^@^@^ @^P^@^@)^@^@^@^A ^@^@^@<FE><FF> <FF><FF>^@^@^@^@ &^@^@^@<FF><FF> <FF>
            \stoptyping}
          } {Binary Format} 
       \stopcombination
      }
     }

A text editor is a tool for editing files in \keyword{plain text}. The is the simplest encoding of text; we're not permitted to embed any presentation codes (for bold or italic fonts, etc.) or other structural features (footnotes, page numbers, etc.) into the format.  The only thing the formats supports is text, plain text: a sequence of the characters and whitespace.

We need to create our HTML and CSS files in plain text because that is \keyword{what the web server is expecting} when it uses our files to render and present our web pages. If we create our HTML and CSS files in, say, Microsoft Word, we'll end up with a bunch of extra information (formatting codes, etc.) that the web browser won't be able to handle, and our webpages will have a lot of extra junk in them (or maybe won't render at all!). We need a text editor to create our files in plain text.

Brackets is a good choice. It's free to use, so we get to learn a tool and take it with us. Brackets is also built from the ground up to be a editor specifically for web markup, and has several nice features that make working with HTML and CSS more pleasant.

Brackets is a good choice, but not the only choice! There are several other text editors available that are also good: Atom, Sublime Text, BBEdit, Notepad++, and Emacs are all fine text editors. If you know some other text editor well, and would like to use it for this course, that's fine by me.\footnote{Although I may be less equipped to help with any troubleshooting - I don't know every editor all that well.}

\stopsubsection

\startsubsection[title={Google Chrome}]

We will, of course, need to use a web browser to render and view our work. For this course, we'll be using {\em Google Chrome}, a popular and (mostly) standards-compliant web browser available for free for MacOS, Windows, and Linux.

While most modern web browsers render web pages in a uniform way, each browser still has some quirks. In a production environment, we often have to do a lot of extra work to make sure our websites render well on all browsers. To make things simpler, in this course we will be developing web content specifically for Chrome.

\useURL[chrome][https://chrome.google.com/]
Google Chrome is available as a free download from: 

\startnarrower
\from[chrome]
\stopnarrower

\stopsubsection

\startsubsection[title={Adobe Photoshop (or other image manipulation program)}]

{\em Adobe Photoshop} is the industry-standard tool for manipulating raster images. We'll be using Photoshop to manipulate and edit images we'll be using in our websites. If you already have experience with a different photo editor, using that should be fine.

If you do not have Photoshop on your personal machine, Photoshop is available on the lab computers.

\stopsubsection

\startsubsection[title={The Git Version Control System}]

{\em Git} is a version control system. We can use Git to track progress on a project, go back and visit previous versions of the project, and collaborate with our peers without worrying about clobbering each other's work. We won't be using all of Git's features in this lab; mostly, it will be useful as a way for you to manage and submit your work, and for me to look at your work and provide comments.

We'll be using a \keyword{git client} on your machine to track changes in your homework and group projects, share source code, and interact with GitHub, a hosted service for maintaining and managing group projects. The recommended GitHub client for this course is Github Desktop.

\useURL[github][https://www.github.com]
\useURL[githubdesktop][https://desktop.github.com]

The GitHub website is here:

\startnarrower
\from[github]
\stopnarrower

GitHub Desktop is available for Mac and Windows as a free download from: 

\startnarrower
\from[githubdesktop]
\stopnarrower

\stopsubsection

\stopsection

\startsection[title={Setting Up a GitHub Account}]

To set up a GitHub account:

\startitemize[n]
\item Navigate to \from[github]
\item Choose and enter a username (anything you like, but out to be recognizable).
\item Enter an email address (this should be your \code{andrew.cmu.edu} address).
\item Enter a password for your account.
\item Click \uicontrol{Sign up for GitHub} (big green button)
\stopitemize

On the following page ("Step 2: Choose your plan"), you'll be presented with a series of options for paid plans, a chance to set up an organization, and the button for creating your account.

{\it You do {\bi not} need to sign up for a paid plan.} 

GitHub was originally developed as an open resource. You are cheerfully encouraged to use it for free, with the trade off that all of the work you store on GitHub is publicly available for others to peruse and learn from. Paid plans are useful if you want to keep your work private. We'll be using an \keyword{organization} for all classwork: it's good to remember that work stored in your personal GitHub space will be public, while work stored in the class organization will be private.

{\it You do {\bi not} need to set up an organization.}

We already have an organization set up that we will be using for this class; more on that next week.

To finish signing up for GitHub, click \uicontrol{Finish sign up} at the bottom of the page.

\stopsection

\startsection[title={Joining the GitHub Organization for the Lab}]

Once you've signed up for GitHub, use the following link to join the GitHub organization for the lab:

\useURL[githuborglink][https://classroom.github.com/a/7CfBFSDG]
%\useURL[githuborglink][TBD]

\from[githuborglink]

You may need to log in again with your GitHub account.

The first time you log in, you will need to select yourself from the classroom roster list by clicking on your Andrew user id.

Then, you will need to accept the assignment \quote{gradual-website} by clicking on the \uicontrol{Accept this assignment} button.

This will create your repository, and populate it with starter code.

\stopsection

\startsection[title={Plagiarism and Licensing}]

\useURL[plagpolicy][http://www.cmu.edu/academic-integrity/plagiarism/index.html]

Plagiarism is not\footnote{Can not be!} tolerated in this class. 

This includes copying someone else's code and claiming it as your own.  Everything you submit for your assignments, code review, and final project should be your code, developed by you and written from scratch. 

A special case is the use of jQuery plug-ins, pre-packaged modules, and other supporting code freely available on the web. It's okay to use these resources to {\em support} your design, but it's very important that their use does not {\em supplant} your design. Your design should still be able to stand on its own without any supporting materials.

Another special case is the use of unlicensed images. It's okay to use unlicensed images to support your design (it's often difficult to find really good images that show exactly what you want), but make sure that these images are clearly referenced so that they don't end up getting used improperly outside of classwork.

{\bf When you use modules, plug-ins, verbatim code, images, or any other material that you did not create as part of your assignments, always be sure to clearly and completely credit the original author.} 

When developing your assignments, you will likely encounter design problems that have been already been solved by many others (usually in as many different ways!), and their code will be freely available to read and (with attribution) use. I encourage seeking out and reading as many examples of others' solutions as you possibly can: this is useful work for time spent, and seeing different approaches should give you a better idea of the patterns of solutions that are available for you to try. 

You will get the most mileage out of this class with practice, by implementing your own versions of solutions to see how they work. This practice should give you a better understanding of what's happening in the code, and this understanding becomes critical when projects grow in size and complexity. With big projects, simple changes can cause surprising side effects\footnote{Surprising, incredibly {\em frustrating} side effects.}. 

In the long run, a general grasp of this stuff will be far more useful than knowledge of specific drop-in solutions.

For a full discussion of Carnegie Mellon's Academic Integrity Policy, please see:\\ \from[plagpolicy]

\stopsection

\startsection[title={Homework}]

For the first part of this lab, I'll be assigning small pieces of homework from week to week. These assignments are going to be cumulative, each building on the work of the previous week; if you have to miss a lab meeting, please check with your group mates, look at the lecture notes, and/or contact me if you have any questions. As we're building on our work from week to week, keeping up is important to get the most out of lab.

Homework for the first part of class will be graded with a simple check plus/minus system:

% Begin rubric table
\definesymbol [checkmark] [\char"2713]
\setupTABLE[column][1][width=8mm,offset=\dimexpr1mm+2pt,align=flushleft,style=\bfx]
\setupTABLE[column][4][width=52mm]
\setupTABLE[row][1][bottomframe=on]
\bTABLE[split=repeat,option=stretch]% head on every page, stretch columns
%
% IMPORTANT: use \bTH ... \eTH to enclose the head|next cells
\bTABLEhead
\bTR
  \bTH  \eTH
  \bTH  Syntax \eTH
  \bTH  Semantics \eTH
  \bTH  Assignment Requirements \eTH
\eTR
\eTABLEhead
% 
\bTABLEnext % setup for next table head
\bTR
  \bTH  \eTH
  \bTH  Syntax \eTH
  \bTH  Semantics \eTH
  \bTH  Assignment Requirements \eTH
\eTR
\eTABLEnext
%
% the table body (main part)
%
\bTABLEbody
%
\bTR
	\bTC \checkmark + \eTC
	\bTC No errors. \eTC
	\bTC No problems. \eTC
	\bTC No problems. \eTC
\eTR
\bTR
	\bTC \checkmark \eTC
	\bTC Understanding of the HTML/CSS syntax.  No major syntactical errors. \eTC
	\bTC Understanding of semantically meaningful HTML \eTC
	\bTC Meets all the requirements. There may be one or two very minor errors that do not affect the overall assignment. \eTC
\eTR
\bTR
	\bTC \checkmark − \eTC
	\bTC Partial understanding of the HTML/CSS syntax. There are several syntax errors cannot be considered mere careless errors. \eTC
	\bTC Partial understanding of semantically meaningful HTML. There are several tags that are not appropriate for the content. \eTC
	\bTC Partially meets the requirements. \eTC
\eTR
%
\eTABLEbody
%
% the table foot
%
% \bTABLEfoot
% \bTR
%   \bTC  foot1 \eTC
%   \bTC  foot2 \eTC
%   \bTC  foot3 \eTC
% \eTR
% \eTABLEfoot
%
\eTABLE

Please make sure that homework is submitted on time; I want to make sure that I have time to check the work before the next class.

We'll be using Git as the primary way of submitting homework. We'll go into more detail about this next week. If you end up having trouble submitting homework via GitHub, you can always mail it to me:

\startnarrower
\code{pm39@andrew.cmu.edu}
\stopnarrower

...or even print it out and putting a hard copy in my mailbox (Baker Hall) as a last resort. Submitting your homework on time is your responsibility, and late work will be tracked.

\stopsection

\startsection[title={Optional Readings}]

\useURL[introhtml][http://www.w3schools.com/html/html_elements.asp]
\useURL[w3cMetadata][http://www.w3.org/TR/html5/document-metadata.html]
\useURL[lynda-github][http://www.lynda.com/GitHub-tutorials/GitHub-Web-Designers/162276-2.html]

\startitemize
\item \from[introhtml] (read)
\item \from[w3cMetadata] (skim)
\stopitemize

The first reading is a gentle introduction to HTML. If you are unfamiliar with markup languages and how they work, try the examples. 

The second reading is the public technical specification for the kinds of metadata that websites can include as part of their structural markup. {\bf You only need to skim this.}  While this material can be dry, dense, and somewhat difficult to absorb,\footnote{Like any good specification!}  it's good to know where to find the authoritative specification, and it might become useful for solving puzzles\footnote{Puzzles like, "Why in the world is my webpage doing \it{that?}"} as our designs become more complex.

\stopsection

\stopchapter

\stopcomponent
